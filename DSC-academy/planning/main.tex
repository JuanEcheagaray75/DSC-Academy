\documentclass{article}
\usepackage[utf8]{inputenc}
\usepackage[spanish]{babel}
\usepackage[]{amsthm}
\usepackage{amsmath}
\usepackage[]{amssymb}
\usepackage{graphicx}
\usepackage{wrapfig}
\usepackage[letterpaper, margin=1.5in]{geometry}
\usepackage[hidelinks]{hyperref}
\decimalpoint

\title{Planeación}
\author{Juan Pablo Echeagaray González}
\date{1 de julio del 2022}


\begin{document}
    \begin{titlepage}
        \maketitle
    \end{titlepage}

    \tableofcontents
    \clearpage

    \section{Data Science Club Academy}

        \begin{itemize}
            \item Máximo de 60 alumnos en el DSC-Academy
            \item Preparar entrevistas en caso de necesitar un nuevo equipo de coordinadores para fungir como maestros
        \end{itemize}

        \subsection{Fundamentos matemáticos [2 personas]}

            \subsubsection{Introducción}

                \begin{itemize}
                    \item Notación matemática
                    \item Repaso del significado de los símbolos*
                    \item Teoría de conjuntos*
                    \item Métodos de demostración
                    \item Teoría de números (propiedades de los reales, etc)
                    \item Construcción del plano*
                    \item GCD, MCD, primos, coprimos, relaciones de equivalencia*
                    \item Factorización
                \end{itemize}

            \subsubsection{Cálculo introductorio}

                \begin{itemize}
                    \item Límite (épsilon-delta)
                    \item Concepto de diferencial*
                    \item Derivada por definición*
                    \item Integral de Riemann

                    \item Métodos de integración
                    \begin{itemize}
                        \item Cambio de variable
                        \item Por partes
                        \item Sustitución Trigonométrica
                        \item Fracciones Parciales
                        \item Integración numérica
                    \end{itemize}

                    \item Métodos de derivación
                    \begin{itemize}
                        \item Regla de la cadena
                        \item Regla del producto y del cociente
                        \item Derivación implícita
                        \item Derivadas parciales
                    \end{itemize}
                \end{itemize}

            \subsubsection{Álgebra de matrices}

                \begin{itemize}
                    \item Operaciones básicas de matrices
                    \item Sistemas de ecuaciones lineales
                    \item Traspuesta
                    \item Adjunta
                    \item Determinante
                    \item Diagonalización
                    \item Inversa
                \end{itemize}

        \subsection{Álgebra lineal [1 persona]}

            \begin{itemize}
                \item Espacios vectoriales
                \item Sub-espacios
                \item Base, conjunto generador y espacio generado
                \item Independencia lineal
                \item Independencia geométrica
                \item Rango
                \item Ortogonalidad
                \item Kernel
                \item Mapeos lineales
                \item Introducción al cálculo tensorial.**
            \end{itemize}
        
        \subsection{Geometría analítica [1 persona]}

            \begin{itemize}
                \item Normas*
                \item Producto interno*
                \item Métricas*
                \item Espacios métricos*
                \item Base ortonormal
                \item Complemento ortogonal
                \item Producto interno de funciones
                \item Proyecciones
                \item Rotaciones
            \end{itemize}
                
        \subsection{Cálculo Vectorial [1 persona]}

            \begin{itemize}
                \item Diferenciación parcial (en profundidad)*
                \item Derivadas de orden superior*
                \item Gradientes*
                \item Gradientes de func. vectoriales
                \item Gradientes de matrices
                \item Back-propagation
                \item Linealización
            \end{itemize}

        \subsection{Probabilidad y estadística [1 persona]}
            \begin{itemize}
                \item Concepto de probabilidad
                \item Probabilidad condicional
                \item Teorema de Bayes
                \item Distribuciones de probabilidad
                \item Teorema del límite central
                \item Estadística descriptiva
                \item Estadística multi-variada
                \begin{itemize}
                    \item Medidas de tendencia central
                    \item Medidas de dispersión
                \end{itemize}
            \end{itemize}
        
        \subsection{Investigación de operaciones [1 persona]}
            
            \begin{itemize}
                \item Optimización
                \item Programación lineal*
                \item Descenso de gradiente*
                \item Algoritmos de búsqueda*
                \item Optimización probabilística*
            \end{itemize}
        
        \subsection{Algoritmos [2 personas]}
            \begin{itemize}
                \item Regresiones
                \begin{itemize}
                    \item Lineal
                    \item Simple
                    \item Lasso
                    \item Ridge
                    \item Bayesiana
                \end{itemize}

                \item Clasificadores
                \begin{itemize}
                    \item Árboles de decisión
                    \item Regresión logística
                    \item Random forest
                    \item SVM
                    \item TPA
                    \item K-NN
                    \item Naive-Bayes
                \end{itemize}

                \item Reducción de dimensionalidad
                \begin{itemize}
                    \item PCA
                    \item Clustering
                    \item Complejos simpliciales
                \end{itemize}

                \item Clustering
                \begin{itemize}
                    \item K-Means
                    \item Aglomerativo
                    \item Espectral
                    \item Jerárquico
                \end{itemize}
            \end{itemize}
        
        \subsection{Aplicaciones (Programación) [2 personas]}
            
            \begin{itemize}
                \item Introducción a la Ciencia de Datos en Python
                \item Introducción a bases de datos (MySQL, Postgres, etc\dots)
                \item Visualización y representación de datos [Agregar creación de \emph{Dashboards}]
                \item Aplicación de Machine Learning
                \item Minería de texto
                \item Análisis de redes sociales
            \end{itemize}

        \subsection{Soft-Skills [1 persona]}

            \begin{itemize}
                \item Storytelling de datos
                \item Escritura de artículos y reportes
                \item Visualización efectiva de datos
                \item Uso de control de versiones
            \end{itemize}
            
        \subsection{Tiempos tentativos}
        
        Se planea tener reuniones semanales de 4 horas. Queda por determinar el lugar donde estas serán dadas, hay que comunicarse con el maestro asesor para ver como podríamos tener acceso a un salón de clases.

        \begin{itemize}
            \item Fundamentos matemáticos: 2 sesiones
            \item Álgebra lineal: 3 sesiones
            \item Geometría: 1 sesión
            \item Cálculo vectorial: 2 sesiones
            \item Probabilidad y estadística: 3 sesiones
            \item Algoritmos: 4 sesiones
            \item Aplicaciones: Sobre la marcha, dados como actividades en paralelo
            \item Soft-Skills: 2 sesiones
        \end{itemize}

    \section{Equipo competitivo}
        
        \begin{itemize}
            \item Equipo de hasta 5 integrantes
            \item Deben de atender las clases del DSC Academy
            \item Seleccionados por un concurso consistente en una serie de actividades que serán evaluadas por Nemesio y Juan
        \end{itemize}

\end{document}